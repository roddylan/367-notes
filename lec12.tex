\documentclass{article}
\usepackage{textcomp, gensymb}
\usepackage{utf8add}
\usepackage[most]{tcolorbox}
\usepackage{hyperref}
\usepackage{cleveref}
\usepackage{amsmath}
\usepackage{amssymb}
\usepackage{mathtools}
\usepackage{systeme}
\usepackage{tcolorbox}
\usepackage{outlines}
\usepackage{bbm}
\usepackage[ruled,longend]{algorithm2e}

\DeclareMathOperator*{\argmax}{arg\,max}
\DeclareMathOperator*{\argmin}{arg\,min}

\hypersetup{
  colorlinks,
  citecolor=black,
  filecolor=black,
  linkcolor=black,
  urlcolor=black
}


\title{CMPUT 367}
\author{Roderick Lan}
\date{}

\usepackage{natbib}
\makeatletter
% \crefformat{tcb@cnt@Example}{example~#2#1#3}
% \Crefformat{tcb@cnt@Example}{Example~#2#1#3}
\makeatother
\newtcbtheorem[auto counter, number within = subsection]
{definition}{Definition}{%                                                        
  breakable,
  fonttitle = \bfseries,
  colframe = blue!75!black,
  colback = blue!10
}{def}


\makeatother
\newtcbtheorem[auto counter, number within = subsection]
{example}{Example}{%                                                        
  breakable,
  fonttitle = \bfseries,
  colframe = orange!75!black,
  colback = orange!10
}{ex}

\makeatother
\newtcbtheorem[auto counter, number within = subsection]
{thm}{Thm}{%                                                        
  breakable,
  fonttitle = \bfseries,
  colframe = orange!75!black,
  colback = orange!10
}{thm}

\makeatother
\newtcbtheorem[auto counter, number within = subsection]
{expln}{Explain}{%                                                        
  breakable,
  fonttitle = \bfseries,
  colframe = red!75!black,
  colback = red!10
}{exp}

\makeatother
\newtcbtheorem[auto counter, number within = section]
{proof}{Proof}{%                                                        
  breakable,
  fonttitle = \bfseries,
  colframe = gray!75!black,
  colback = gray!10
}{prf}


\begin{document}

\maketitle

\tableofcontents
\break

\section*{Lecture 12 - Feb 29}
% \noindent\rule{\textwidth}{0.5pt}
\section{Neural Networks}
Neuron takes some input $x_1 \cdots x_d$
\begin{flalign*}
    z &= w_1x_1 + \cdots + w_dx_d + b \\
    &= \sum_i w_i x_ i+ b \\
    &= w^\top x + b
\end{flalign*}
\[
    y = f(z) \ \ \ \ \ \ \text{(activation function)}
\]
NN layers usually fully connected.
\\
Input layer - data features
\\
Output layer - prediction
\\
Hidden layers ($\ge 1$)

\noindent\rule{\textwidth}{0.5pt}\\[5pt]
Suppose the weights (params) of a neural network is known. \\
How can we compute the output of the neural network?

\subsection{Recursion/Iteration Process (Forward Prop)}
\begin{enumerate}
    \item \textbf{Initialization}
        \begin{outline}
            \1 first layer is simply features
        \end{outline}
    \item \textbf{Recursion Step}
    \begin{flalign*}
        \intertext{Assume $y^{(l-1)}$ is known; Calculate $y^{(l)}$}
        z_i^{(l)} &= \sum_{j=1}^{d^{(l-1)}} w_{ij}^{(l)} y_j ^{(l-1)} + b_i ^{(l)} \\
        y_i^{(l)} &= f(z_i^{(l)})
    \end{flalign*}
    \item \textbf{Termination}
    \begin{outline}
        \1 when the output layer's value is computed
    \end{outline}
\end{enumerate}
Matrix Vector:
\begin{flalign*}
    \mathbf z^{(l)} &= \mathbf W ^{(l)} \mathbf y ^{(l-1)} + \mathbf b^{(l)}
    \\
    \mathbf y^{(l)} &= f(\mathbf z ^{(l)})
\end{flalign*}
\[
    \mathbf Y = \begin{bmatrix}
        \mathbf y^{(1)} & \mathbf y^{(2)} & \cdots & \mathbf y^{(m)}
    \end{bmatrix}^\top
\]
\[
    \mathbf Z^{(l)} = \mathbf  Y^{(l-1)} \mathbf W^{(l)} + \begin{bmatrix}
        - & \mathbf b^\top & -\\
        - & \mathbf b^\top & -
        \\ & \vdots & \\
        - & \mathbf b^\top & -
    \end{bmatrix}
\]
% \[
%     \mathbf z ^{(l)}= \mathbf W^{(l)} \mathbf y^{(l-1)} + \mathbf b^{(l)}
% \]

\subsection{Train Weights}
$\theta = (w^{(1)}, b^{(1)}, \cdots, w^{(L)}, b^{(L)})$
\[
    \theta ^{(\text{new})} \gets \theta ^{(\text{old})} - \nabla_\theta J(\theta^{(\text{old})})
\]
Basic Idea: Apply chain rule top-down
\subsection{Recursive Process (Backprop)}
\begin{enumerate}
    \item \textbf{Initialization}
    \item \textbf{Recursion}
    \item \textbf{Termination}
\end{enumerate}

























































\end{document}