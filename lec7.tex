\documentclass{article}
\usepackage{textcomp, gensymb}
\usepackage{utf8add}
\usepackage[most]{tcolorbox}
\usepackage{hyperref}
\usepackage{cleveref}
\usepackage{amsmath}
\usepackage{amssymb}
\usepackage{mathtools}
\usepackage{systeme}
\usepackage{tcolorbox}
\usepackage{outlines}
\usepackage[ruled,longend]{algorithm2e}

\DeclareMathOperator*{\argmax}{arg\,max}
\DeclareMathOperator*{\argmin}{arg\,min}

\hypersetup{
  colorlinks,
  citecolor=black,
  filecolor=black,
  linkcolor=black,
  urlcolor=black
}


\title{CMPUT 367}
\author{Roderick Lan}
\date{}

\usepackage{natbib}
\makeatletter
% \crefformat{tcb@cnt@Example}{example~#2#1#3}
% \Crefformat{tcb@cnt@Example}{Example~#2#1#3}
\makeatother
\newtcbtheorem[auto counter, number within = subsection]
{definition}{Definition}{%                                                        
  breakable,
  fonttitle = \bfseries,
  colframe = blue!75!black,
  colback = blue!10
}{def}


\makeatother
\newtcbtheorem[auto counter, number within = subsection]
{example}{Example}{%                                                        
  breakable,
  fonttitle = \bfseries,
  colframe = orange!75!black,
  colback = orange!10
}{ex}

\makeatother
\newtcbtheorem[auto counter, number within = subsection]
{thm}{Thm}{%                                                        
  breakable,
  fonttitle = \bfseries,
  colframe = orange!75!black,
  colback = orange!10
}{thm}

\makeatother
\newtcbtheorem[auto counter, number within = subsection]
{expln}{Explain}{%                                                        
  breakable,
  fonttitle = \bfseries,
  colframe = red!75!black,
  colback = red!10
}{exp}

\makeatother
\newtcbtheorem[auto counter, number within = section]
{proof}{Proof}{%                                                        
  breakable,
  fonttitle = \bfseries,
  colframe = gray!75!black,
  colback = gray!10
}{prf}


\begin{document}

\maketitle

\tableofcontents
\break

\section*{Lecture 7 - Feb 1}
\section{Convexity Cont'd}
\begin{thm}
{}{}
Let $f$ be a twice differentiable function on a convex domain;
\\
the following three conditions are equivalent
\begin{enumerate}
    \item $f$ is a convex function
    \item {[1st Order]} $\forall x,y \in \mathrm{dom}f$, 
    \[
        f(y) \ge f(x) + [\nabla f(x)]^\top (y-x)
    \]
    (Linear approx is always a lower bound)
    \item {{2nd order}} $\forall x \in \mathrm{dom} f$ \[
        \nabla ^2 f(x) \ge 0
    \]
    (positive semidefinite)
\end{enumerate}
\end{thm}
\noindent
1D: $f''(x) \ge 0$ then curve is convex\\
High-D: 
\begin{list}{}{}
    \item if $\nabla^2 f(x)$ diagonal - eigenval are diagonal elements;\\
        nonnegative evals $\to$ every dimension is curving up $\to$ convex
    \item $\nabla^2 f(x)$ not diagonal - check eigenvals; \\negative eval $\to$ not convex 
    \\
    nonnegative eval only $\to$ convex
\end{list}
\pagebreak
\begin{proof}
    {2. $\to$ 1.}{}
    $\forall x,y \in \mathrm{dom} f; \ \ \forall \lambda \in (0,1)$\\
    The first order condition:
    \[
        \forall x,y \in \mathrm{dom}f: f(y) \ge f(x)+(\nabla_x f(x))^\top (y-x)
    \]
    Put $z = \lambda x + (1-\lambda)y$ into the $x$ 
    \begin{equation}
        f(y) \ge f(z) + [\nabla_z f(z)]^\top (y-z)
    \end{equation}
    Put $z$ into $x$ and $x$ into $y$
    \begin{equation}
        f(x) \ge f(z) + [\nabla_z f(z)]^\top (x-z)
    \end{equation}
    $\lambda (2) + (1-\lambda)(1)$ gives us:
    \begin{flalign*}
        \lambda f(x) + (1-\lambda)f(y) &\ge \lambda f(z) + (1-\lambda) f(z)\\
        &+ (1-\lambda) [\nabla_z f(z)]^\top(y-z) + \lambda[\nabla _z f(z)]^\top (x-z)\\
        % 
        &\ge f(z)
        + [\nabla_zf(z)]^\top (y-z - \lambda y + \lambda z +\lambda x - \lambda z )\\
        &= f(z)
        + [\nabla_zf(z)]^\top [\lambda x + (1-\lambda)y - z]
        \intertext{Since $z = \lambda x + (1-\lambda)y$ (from (1))}
        &= f(z) + [\nabla_zf(z)]^\top [0]  = f(z)
    \end{flalign*}
\end{proof}
\begin{proof}
    {3. $\to$ 2.}{}
    $\forall x,y \in \mathrm{dom}f$
    \[
        f(y) = f(x) + [\nabla f(x)]^\top (y-x) + \frac{1}{2}(y-x)^\top [\nabla^2 f(z)](y-x)
    \]
    for some $z$ between $x$ and $y$.\\
    Taylor Expansion to first order:
    \[
        T(y) = f(x) + [\nabla f(z)]^\top (y-x) \text{ for $z$}
    \]
    (MVT)
    Since $\nabla^2 f(z)\ge 0$, second order term $\ge 0$; Thus
    \[
        f(y) \ge T(y)
    \]
\end{proof}

\section{MSE}
\begin{flalign*}
    \nabla_w J_{\mathrm{MSE}} &= \frac{1}{M}(X^\top X w - Xt)\\
    \nabla w^2_w J_{\mathrm{MSE}} &= \frac{1}{M} X^\top X
    \intertext{$\forall v, v^\top X^\top X v = (Xv)^\top Xv = \| Xv \|_2^2 \ge 0$}
\end{flalign*}

\begin{definition}
    {Optimality}{}
    $x$ is a \textbf{Global Optimum} for $f$ iff 
    \[
        \forall y\in \mathrm{dom}f, \ f(x) \le f(y)
    \]
    $x$ is a \textbf{Local Optimum} of $f$ iff
    \[
        \exists \epsilon > 0 \text{ s.t. } \forall y\in \mathrm{dom}f
    \]
    if $\| x- y \| < \epsilon$ then $f(x) \le f(y)$
\end{definition}


\begin{thm}
    {}{}
    Let $f$ be a convex function, a local optimum $x$ of $f$ is a global optimum
\end{thm}






























































\end{document}